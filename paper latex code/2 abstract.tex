
\begin{abstract}
%% AIMS: What practical or theoretical problem does the research respond to, or what research question did you aim to answer?
Proprioception, or the perception of the configuration of one's body, is challenging to achieve with soft robots due to their infinite degrees of freedom and incompatibility with most off-the-shelf sensors. 
%
This work explores the use of inertial measurement units (IMUs), sensors that output orientation with respect to the direction of gravity, to achieve soft robot proprioception.
%
A simple method for estimating the shape of a soft continuum robot arm from IMUs mounted along the arm is presented.
The approach approximates a soft arm as a serial chain of rigid links, where the orientation of each link is given by the output of an IMU or by spherical linear interpolation of the output of adjacent IMUs.
%
In experiments conducted on a 660mm long real-world soft arm, this approach provided estimates of its end effector position with a median error of less than 10\% of the arm's length.
%
This demonstrates the potential of IMUs to serve as inexpensive off-the-shelf sensors for soft robot proprioception.


\end{abstract}

\begin{IEEEkeywords}
Modeling, Control, and Learning for Soft Robots; Kinematics; Sensor Fusion
\end{IEEEkeywords}