\section{Discussion} \label{sec:discussion}

% \subsection{Likely causes of error and openings for improvement}    %% Don't need subsections here
A likely cause of error for the rigid-body model is the length of the segments for high bending values. 
Bending reduces the distance between the joints, whereas the rigid-body model keeps it constant.
%
The PCC-extended rigid-body model captures the overall shape better, but since the model is a serial chain, a small error near the base propagates through the entire arm.
This is why the error at the end-effector tends to be larger than that at the motion capture marker near the middle of the arm (see Fig.~\ref{fig:hull}).
% Future work could investigate using non-uniform distributions of IMUs to increase sensitivity in regions with the most expected bending.

Another likely source of error is the orientation of the arm relative to gravity.
In our experiments the arm was mounted such that it points upwards from the base, causing the weight of the arm to induce larger non-constant curvatures near the base.
This curvature near the base leads to errors that propagate and grow along the length of the arm.
% The heads up position of the arm tends to show non-PCC behavior of the arm, which would explain the error. 
We could potentially reduce this error by mounting the arm such that it points downward instead, 
% it the arm heads down would probably increase highly the accuracy of the system, typically in an industrial context.
or by changing the IMU positions along the arm such that the density of IMUs increases as we get closer to the arm's base.

IMU drift might also generate errors in the IMU measurements, especially at the base. 
The IMU we used, the bno055, auto-calibrates in the background by reaching different positions \cite{sensortec2016bno055}.
The first IMUs along the arm have a limited range of positions, which might reduce the quality of their calibration and therefore increase the error.

Ideally, the shape of the arm would be computed from IMU data quickly enough to provide a feedback signal to a real-time closed-loop controller.
In our experiments the rigid-body model was able to provide estimates in real-time, but the PCC-extended rigid-body model was not.
Because this work is meant only to be a proof of concept for IMU based proprioception, achieving real-time computation was not our focus.
However, there are many ways in which computational efficiency could be improved in the future.
% The study is meant to be a proof of concept: though, computational efficiency could be improved to have PCC-Extended Rigid-Body model running online. 
One such approach could be to ignore the real part of the quaternions and keep only a vector along which the arm would be aligned. 
Then linear interpolation could be used between these vectors, which is computationally 3 times lighter than spherical linear interpolation using MATLAB implementation \cite{matlab}. 


% \subsection{How many IMUs are needed for a given arm with the Rigid-body model?}     %% Don't need subsections here

Fig.~\ref{fig:err_pcc16_model} shows that increasing the number of actual IMU sensors used to generate a 16 segment model reduces the model's error. However, Fig. \ref{fig:plateau} shows that increasing the number of segments beyond 16 does not noticeably reduce the error. We interpret this to mean that a 16 segment rigid-body model is a suitable approximation of a PCC model of this arm.
% Therefore, we expect that 16 is the maximum number of IMUs needed to approximate the shape of the arm as accurately as a PCC model.
Therefore, we expect that for 16 or more IMUs, a rigid-body model and a continuous PCC model will approximate the shape of the arm with roughly equal accuracy.
This ``maximum number of IMUs'' will vary from arm to arm depending on its geometry and bending stiffness.
Future work could develop a method to compute this number for any soft arm as a function of its maximum curvature and some tolerated error threshold.
There are computation concerns when adding additional IMUs, however. Adding more IMUs could reduce the reading frequency, especially if using a Micro Controller Unit. This could be compensated by using higher speed protocols than I2C such as SPI.

%% Old version
% This sufficient approximation number of segments value could be used as the number of IMUs that should be used to estimate the state of a given arm at best, for instance using the rigid-body model. \Yves{I don't like this sentence, we should explain better why if you can't reduce the error with more segments, then why would you reduce the error with more IMUs?}
% . 
% This method could also be generalized to compute the number of required segments to model any soft arm, w.r.t its maximum curvature and the tolerated error in future work.






